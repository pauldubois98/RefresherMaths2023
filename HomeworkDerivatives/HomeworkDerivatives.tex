\documentclass[]{article}

%format
\usepackage[utf8]{inputenc}
\usepackage[T1]{fontenc}
\usepackage[english]{babel}
\usepackage[margin=2.5cm]{geometry}
%math
\usepackage{amsthm}
\usepackage{amsmath}
\usepackage{amsfonts}
\usepackage{amssymb}
\usepackage{stmaryrd}
\usepackage{nicefrac}
\usepackage{mathtools}
%others
\usepackage{hyperref}
\usepackage{graphicx}
\usepackage{enumitem}

%environments
\newtheorem{question}{Question}
%commands
%\newcommand{\name}[num]{definition}
\newcommand{\primes}{\mathbb{P}}
%\newcommand{\P}{\mathbb{P}}
\newcommand{\N}{\mathbb{N}}
\newcommand{\Z}{\mathbb{Z}}
\newcommand{\Q}{\mathbb{Q}}
\newcommand{\D}{\mathbb{D}}
\newcommand{\R}{\mathbb{R}}
\newcommand{\C}{\mathbb{C}}
\newcommand{\F}{\mathbb{F}}
\newcommand{\B}{\mathbb{B}}
\newcommand{\Norm}[2][]{\text{Norm}_{#1}(#2)}
\newcommand{\norm}[2][]{\text{Norm}_{#1}(#2)}
\newcommand{\inner}[2]{\left\langle #1,#2 \right\rangle}
\newcommand{\floor}[1]{\lfloor #1 \rfloor}
\newcommand{\ceil}[1]{\lceil #1 \rceil}
\newcommand{\abs}[1]{| #1 |}
\newcommand{\card}[1]{| #1 |}
\newcommand{\curt}[1]{\sqrt[3]{#1}}
\newcommand{\Ker}[1]{\text{Ker}(#1)}
\newcommand{\Image}[1]{\text{Im}(#1)}
\newcommand{\Trace}[1]{\text{Tr}(#1)}
\newcommand{\Det}[1]{\text{Det}(#1)}
\newcommand{\degree}[1]{\partial #1}
\newcommand{\Pow}[1]{\mathcal{P}(#1)}

%opening
\title{Derivatives}
\author{}
\date{}

\begin{document}
	
	\maketitle
	
	
	\begin{question}
		Calculate the derivative of the following functions:\\
		\begin{itemize}
			\item $f_0(x) = 3x^2$
			\item $f_1(x) = 5x^2-18$
			\item $f_2(x) = 5x^2-18x+39$
			\item $f_3(x) = \sin(x)$
			\item $f_4(x) = \sin(x)*x^2$
			\item $f_5(x) = \frac{5x^3-2x+1}{2x-7}$
			\item $f_6(x) = ax^2+bx+c$
		\end{itemize}
	\end{question}
	
	\begin{question}
		Calculate the second order derivative of the same functions:\\
		\begin{itemize}
			\item $f_0(x) = 3x^2$
			\item $f_1(x) = 5x^2-18$
			\item $f_2(x) = 5x^2-18x+39$
			\item $f_3(x) = \sin(x)$
			\item $f_4(x) = \sin(x)*x^2$
			\item $f_5(x) = \frac{5x^3-2x+1}{2x-7}$
			\item $f_6(x) = ax^2+bx+c$
		\end{itemize}
	\end{question}
	
	\begin{question}
		Find the anti-derivative of the following functions:\\
		\begin{itemize}
			\item $g_0(x) = 3x^2$
			\item $g_1(x) = 5x^2-18$
			\item $g_2(x) = 5x^2-18x+39$
			\item $g_3(x) = \sin(x)$
			\item $g_4(x) = ax^2+bx+c$
		\end{itemize}
	\end{question}
	
	\begin{question}
		Calculate the following partial derivatives:\\
		\begin{itemize}
			\item $h_1(x,y) = 3x^2+y^2$ w.r.t. $x$ ($\frac{\partial h_1}{\partial x}$)
			\item $h_1(x,y) = 3x^2+y^2$ w.r.t. $y$ ($\frac{\partial h_1}{\partial x}$)
			\item $h_2(x,y,z) = 5x^3-18y^2-18x+39z^5+40xy+z^2x^3y$ w.r.t. $x$ ($\frac{\partial h_2}{\partial x}$)
			\item $h_2(x,y,z) = 5x^3-18y^2-18x+39z^5+40xy+z^2x^3y$ w.r.t. $y$ ($\frac{\partial h_2}{\partial y}$)
			\item $h_2(x,y,z) = 5x^3-18y^2-18x+39z^5+40xy+z^2x^3y$ w.r.t. $z$ ($\frac{\partial h_2}{\partial z}$)
		\end{itemize}
	\end{question}
	
	\begin{question}
		Calculate the following second / third order partial derivatives:\\
		\begin{itemize}
			\item $h_1(x,y) = 3x^2+y^2$ w.r.t. $x$ then $y$ ($\frac{\partial^2 h_1}{\partial x \partial y}$)
			\item $h_1(x,y) = 3x^2+y^2$ w.r.t. $y$ then $x$ ($\frac{\partial^2 h_1}{\partial y \partial x}$)
			\item $h_2(x,y,z) = 5x^3-18y^2-18x+39z^5+40xy+z^2x^3y$ w.r.t. $x$ and $x$ ($\frac{\partial^2 h_2}{\partial x^2}$)
			\item $h_2(x,y,z) = 5x^3-18y^2-18x+39z^5+40xy+z^2x^3y$ w.r.t. $y$ and $x$ ($\frac{\partial^2 h_2}{\partial y \partial x}$)
			\item $h_2(x,y,z) = 5x^3-18y^2-18x+39z^5+40xy+z^2x^3y$ w.r.t. $z$ then $x$ and $y$ ($\frac{\partial^3 h_2}{\partial x \partial y \partial z}$)
		\end{itemize}
	\end{question}
	
	\begin{question}
		Calculate 5 steps of gradient descent with learning rate of $\lambda = 0.8$, starting from $x_0 = -0.25$ for the function $f(x) = x^2-x+3$.\\
		Conjecture what is the exact minimum of $f$; how far is $x_1$ from it? and $x_4$?\\
		What happens if the learning rate is $\lambda = 1$? and $\lambda = 2$? and $\lambda = 0.1$? and $\lambda = 0.01$? (do only 3 steps)
	\end{question}
	
	
\end{document}
