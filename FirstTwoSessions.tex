\documentclass[a4paper,12pt]{article}
\usepackage[utf8]{inputenc}
\usepackage{graphicx}
\usepackage{hyperref}
\providecommand{\tightlist}{\setlength{\itemsep}{0pt}\setlength{\parskip}{0pt}}


\usepackage{amsmath}
\usepackage{amsfonts}
\usepackage{amsthm}
\usepackage{amssymb}
\usepackage{mathrsfs}
\usepackage{stmaryrd}
\usepackage{xcolor}
\usepackage{soul}

\newcommand{\Q}{\mathbb{Q}}
\newcommand{\N}{\mathbb{N}}
\newcommand{\Z}{\mathbb{Z}}
\newcommand{\R}{\mathbb{R}}
\newcommand{\Primes}{\mathbb{P}}
\newcommand{\txtand}{\text{ and }}
\newcommand{\txtor}{\text{ or }}
\newcommand{\lxor}{\veebar}

\begin{document}
	
	\author{Paul Dubois}
	\title{Mathematics Refresher Course\\First Two Sessions}
	\date{September 2023}
	
	\maketitle
	
	\begin{abstract}
		This course teaches basic mathematical methodologies for proofs.
		It is intended for students with a lack of mathematical background, or with a lack of confidence in mathematics.
		We will try to cover most of the prerequisites of the courses in the master's, i.e. basic algebra/analysis and basic applications.
	\end{abstract}
	
	\section{Presentation}
	
	\begin{itemize}
		\tightlist
		\item Paul Dubois
		\item 3rd year PhD @ Centrale / TheraPanacea
		\item Research topic: AI applied to radiotherapy
		\item Email:\href{mailto:b00795695@essec.edu}{\nolinkurl{b00795695@essec.edu}} (for any question)
		\paragraph{Course structure}
		\begin{itemize}
			\item 8*3h arranged as 1h20min lecture - 1/3h break - 1h20min lecture
			\item \textcolor{red}{\st{No pb class planned, but lectures will have integrated live
					exercises}}
			\item \textcolor{red}{\st{Interrupt if needed (do \emph{not} wait for the end of the lecture)}}
			\item In this document, you will find the content of the first two sessions, with the small exercises we did "live".
			\item The remaining six sessions will be problem solving.\\
			In case a session is spent on a topic you already, you can skip it \textbf{on the condition that you submit all compulsory exercises corresponding to that session}.
		\end{itemize}
		\item Examination
		\begin{itemize}
			\tightlist
			\item The course is pass/fail
			\item Spoiler: All of you will pass
			\item \textcolor{red}{\st{Home exercises, you will need 80+\% to pass}}
			\item \textcolor{red}{\st{to complete exercises, it should take 30min to
			1h}}
			\item \textcolor{red}{\st{2-4 exercises}}
			\item \textcolor{red}{\st{Hand in paper of PDF}}
			\item \textcolor{red}{\st{In the unlikely event of not passing, you will be able to do some
			extra work to pass}}
			\item To pass, I will ask you, for each session, to either be in class, or submit the compulsory exercises.
			\item The submission deadlines for the exercises set is exactly one week after the corresponding class.
		\end{itemize}
		\item Submitting
		\begin{enumerate}
			\item Solve exercises
			\item Export you work to a single PDF file (e.g. using a scanning smartphone app)
			\item Rename your file "submission\_{nb}\_{family\_name}.pdf" where:
			\begin{itemize}
				\item "nb" is "2" for exercises set 2, "3" for exercise set 3, etc...
				\item "family\_name" is your family name in latin alphabet, capital letters
			\end{itemize}
			\textit{Example: if I wanted to submit exercise set 1, the name of my file should have been "\textbf{submission\_1\_DUBOIS.pdf}"}
			\item Send me one new email per submission, please do not use the "reply" button, create a new email;\\For the subject, you can just put the name of the file (or anything else that makes sense).
		\end{enumerate}
	\end{itemize}
	
	
	
	
	
	
	\section{Sets}
	
	\begin{itemize}
	\tightlist
	\item
	sets of numbers (\(\N\), \(\Z\), \(\R\), \(\Q\), \(\Primes\))
	\item
	complex sets (with \(\{ \}\))
	\item
	examples (draw them):
	
	\begin{itemize}
		\tightlist
		\item
		\(\{n \mid 4<n<10, n \in \N \}\)
		\item
		\(\{2n-1 \mid 4<n<10 , n \in \N \}\)
		\item
		\(\{x \mid 4<x<10, x \in \R \}\)
		\item
		\(\{x \mid 4<x^2<10 \}\)
		\item
		\(\{(x,y) \mid 0<x<2 , 1<y<3, x \in \R, y \in \R \}\)
	\end{itemize}
	\item
	live exercises: draw set + define set from drawing
	\item
	intervals (\(\left[a,b\right]\) \& \(\left(a,b\right)\)); example:
	\(\left[-2, 3\right)\)
	\item
	sets unions \& intersections
	\item
	examples:
	
	\begin{itemize}
		\tightlist
		\item
		\(\left[0,1\right) \cup \left(2,3\right]\)
		\item
		\(\left(0,1\right) \cap \left[0.5,2\right]\)
		\item
		\(\left[-2,5\right) \cap \N\)
		\item
		\(\left[-2,5\right) \cap \Z\)
	\end{itemize}
	\item
	live exercises:
	
	\begin{itemize}
		\tightlist
		\item
		compute and plot the inersection and union of
		\(A = \left(1, 5\right)\) and \(B = \left(3, 7\right]\).
		\item
		compute and plot the inersection and union of
		\(C = \left(-\infty, 2\right]\) and
		\(D = \left[0, +\infty \right)\).
	\end{itemize}
	\item
	quantifiers: \(\forall\), \(\exists\)
	\item
	simple example: $S = \{ 1,3,5,7,8 \} \textit{ ; } \forall s \in S, s \leq 10$
	\item
	example (combined): "for any number, there is a (natural) number
	greater" (\(\forall x \in \R, \exists n \in \N s.t. n>x\))
	\item
	live exercises:
	
	\begin{itemize}
		\tightlist
		\item
		\(S = \{5,6,3,1\}\) "all elements of \(S\) are positive"
		\item
		\(S = \{5,6,3,1\}\) "there is an odd element in \(S\)"
		\item
		\(S = \{5,6,3,1\}\) "there is an even element in \(S\) that is not a
		multiple of 4"
	\end{itemize}
	\item
	implications \(\implies\), \(\impliedby\), \(\iff\)
	\item
	examples:
	
	\begin{itemize}
		\tightlist
		\item
		\(x>1 \implies x \text{positive}\)
		\item
		\(k \in \Z \impliedby k \in \N\)
		\item
		\(k \in \Z  \txtand k\geq 0 \iff k \in \N\)
	\end{itemize}
	\item
	live exercises:
	
	\begin{itemize}
		\tightlist
		\item
		"if \(x\) is positive, then it is the square of another number"
		\item
		"\(n\) is pair is equivalent to \(n=2m\) for some integer \(m\)"
	\end{itemize}
	\item
	extreme values (\(\min\),\(\max\) vs \(\inf\),\(\sup\))
	\item
	live exercises:
	
	\begin{itemize}
		\tightlist
		\item
		find the extreme values of the set \(A = \{x \in \R \mid x>0\}\).
		\item
		find the extreme values of the set
		\(B = \{1-\frac{1}{n} \mid n\in \N\}\).
	\end{itemize}
	\end{itemize}
	
	
	
	
	
	
	\section{Boolean Algebra}
	
	\begin{itemize}
	\tightlist
	\item
	principle (only \(0\) and \(1\))
	\item
	\(+\) and \(*\) for booleans: \(\lor\) and \(\land\)
	\item
	\(not\) (\(\lnot\))
	\item
	tables
	\item
	De Morgan\textquotesingle s law
	(\(\lnot (a \land b) = \lnot a \lor \lnot b\) and
	\(\lnot (a \lor b) = \lnot a \land \lnot b\))
	\item
	\(implications\) operators (\(\implies, \impliedby, \iff\));
	\(\text{x}or\) operator (\(\lxor\))
	\item
	live exercise:
	
	\begin{itemize}
		\tightlist
		\item
		express \(\lxor\) in terms of \(\lor, \land, \lnot\)
		\item
		express \(\implies\) in terms of \(\lor, \land, \lnot\)
		\item
		express \(\land\) in terms of \(\lor, \lnot\)
		\item
		express \(\lor\) in terms of \(\land, \lnot\)
	\end{itemize}
	\end{itemize}
	
	
	
	
	
	
	\section{Modular arithmetic}
	
	\begin{itemize}
	\tightlist
	\item
	Euclidean division of \(a\) by \(b\) (\(a=bk+r\) with
	\(0 \leq r < b\))
	\item
	example with \(a=35\), \(b=2,3,4,5,6,7,8\)
	\item
	modular classes
	(\(12 \equiv 7 \equiv 22 \equiv 102 \equiv -3 \equiv -103 \mod 5\)
	i.e. \(\{2+5k \mid k \in \Z \}\))
	\item
	live exercises:
	
	\begin{itemize}
		\tightlist
		\item
		give 3 numbers that are congruent to 3 mod 7
		\item
		give a test in terms of modular arithmetic that is equivalent to
		"\(n\) is odd"
		\item
		give a test in terms of modular arithmetic that is equivalent to
		"\(n\) is a nultiple of \(k\)" (for \(k\) a natural number greater
		than two)
		\item
		what does it mean for \(n\) to say that \(n \equiv 5 \mod 10\)?
		\item
		find the least positive value of \(x\) such that
		\(71 \equiv x \mod 8\)
	\end{itemize}
	\item
	modular operations (+,-,* \(\mod n\))
	\item
	GCD and \(\square^{-1} \mod p\)
	\item
	example:
	
	\begin{itemize}
		\tightlist
		\item
		compute the GCD of \(270\) and \(192\) (answer: \(6\))
		\item
		compute \(5^{-1} \mod 11\)
	\end{itemize}
	\item
	live exercises:
	
	\begin{itemize}
		\tightlist
		\item
		find the least positive value of \(x\) such that
		\(89 \equiv (x + 3) \mod 4\)
		\item
		what is \(x \mod 10\) if \(96 \equiv x / 7 \mod 5\)
		\item
		find an \(x\) such that \(5x \equiv 4 \mod 11\)
		\item
		if \(x\) is congruent to \(13 \mod 17\) then \(7x - 3\) is congruent
		to which number \(\mod 17\)?
	\end{itemize}
	\end{itemize}
	
	
	
	
	
	
	\section{Functions}
	
	\begin{itemize}
	\tightlist
	\item functions def
	\item image vs pre-image
	\item injective vs surjective
	\item example of a function injective + proof it is
	\item example of a function surjective + proof it is
	\item example of a function not injective + proof it is not
	\item example of a function not surjective + proof it is not
	\end{itemize}
	
	
	
	
	
	\section{Counting Arrangements}
	\paragraph{People in a company}
	In a company, there are 800 employees.\\
	300 are men, 352 are union members, 424 are married, 188 are union men, 166 are married men, 208 are union members and married, 144 are married union men.\\
	How many single, non-union women are there?
	
	\paragraph{The padlock}
	A padlock has a 4-digit code, each number being a number from 0 to 9.
	\begin{enumerate}
		\item How many possible codes are there?
		\item How many possible codes are there with 4 different digits?
		\item How many codes ending in an even number are there?
		\item How many codes are there ending with an even number and with 4 different numbers?
	\end{enumerate}
	In each of the situations:
	\begin{itemize}
		\item How many codes are there containing at least one digit 4?
		\item How many codes are there containing exactly one digit 4?
	\end{itemize}
	
	
	
	
	
	
	
	
\end{document}