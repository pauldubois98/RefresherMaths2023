\documentclass[a4paper,12pt]{article}
\usepackage[utf8]{inputenc}
\usepackage{graphicx}
\usepackage{hyperref}
\providecommand{\tightlist}{\setlength{\itemsep}{0pt}\setlength{\parskip}{0pt}}


\usepackage{amsmath}
\usepackage{amsfonts}
\usepackage{amsthm}
\usepackage{amssymb}
\usepackage{mathrsfs}
\usepackage{stmaryrd}
\usepackage{xcolor}
\usepackage{soul}

\newcommand{\Q}{\mathbb{Q}}
\newcommand{\N}{\mathbb{N}}
\newcommand{\Z}{\mathbb{Z}}
\newcommand{\R}{\mathbb{R}}
\newcommand{\Primes}{\mathbb{P}}
\newcommand{\txtand}{\text{ and }}
\newcommand{\txtor}{\text{ or }}
\newcommand{\lxor}{\veebar}

\begin{document}
	
	\author{Paul Dubois}
	\title{Mathematics Refresher Course\\First Two Sessions}
	\date{September 2023}
	
	\maketitle
	
	\begin{abstract}
		This course teaches basic mathematical methodologies for proofs.
		It is intended for students with a lack of mathematical background, or with a lack of confidence in mathematics.
		We will try to cover most of the prerequisites of the courses in the master's, i.e. basic algebra/analysis and basic applications.
	\end{abstract}
	
	\section{Presentation}
	
	\begin{itemize}
		\tightlist
		\item Paul Dubois
		\item 3rd year PhD @ Centrale / TheraPanacea
		\item Research topic: AI applied to radiotherapy
		\item Email:\href{mailto:b00795695@essec.edu}{\nolinkurl{b00795695@essec.edu}} (for any question)
		\paragraph{Course structure}
		\begin{itemize}
			\item 8*3h arranged as 1h20min lecture - 1/3h break - 1h20min lecture
			\item \textcolor{red}{\st{No pb class planned, but lectures will have integrated live
					exercises}}
			\item \textcolor{red}{\st{Interrupt if needed (do \emph{not} wait for the end of the lecture)}}
			\item In this document, you will find the content of the first two sessions, with the small exercises we did "live".
			\item The remaining six sessions will be problem solving.\\
			In case a session is spent on a topic you already, you can skip it \textbf{on the condition that you submit all compulsory exercises corresponding to that session}.
		\end{itemize}
		\item Examination
		\begin{itemize}
			\tightlist
			\item The course is pass/fail
			\item Spoiler: All of you will pass
			\item \textcolor{red}{\st{Home exercises, you will need 80+\% to pass}}
			\item \textcolor{red}{\st{to complete exercises, it should take 30min to
			1h}}
			\item \textcolor{red}{\st{2-4 exercises}}
			\item \textcolor{red}{\st{Hand in paper of PDF}}
			\item \textcolor{red}{\st{In the unlikely event of not passing, you will be able to do some
			extra work to pass}}
			\item To pass, I will ask you, for each session, to either be in class, or submit the compulsory exercises.
			\item The submission deadlines for the exercises set is exactly one week after the corresponding class.
		\end{itemize}
		\item Submitting
		\begin{enumerate}
			\item Solve exercises
			\item Export you work to a single PDF file (e.g. using a scanning smartphone app)
			\item Rename your file "submission\_{nb}\_{family\_name}.pdf" where:
			\begin{itemize}
				\item "nb" is "2" for exercises set 2, "3" for exercise set 3, etc...
				\item "family\_name" is your family name in latin alphabet, capital letters
			\end{itemize}
			\textit{Example: if I wanted to submit exercise set 1, the name of my file should have been "\textbf{submission\_1\_DUBOIS.pdf}"}
			\item Send me one new email per submission, please do not use the "reply" button, create a new email;\\For the subject, you can just put the name of the file (or anything else that makes sense).
		\end{enumerate}
	\end{itemize}
	
	
	
\end{document}