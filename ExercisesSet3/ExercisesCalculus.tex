\documentclass[]{article}

%opening
\title{Exercises Set 3}
\author{Paul Dubois}

\usepackage{amsmath}
\usepackage{amsfonts}
\usepackage{amsthm}
\usepackage{amssymb}
\usepackage{mathrsfs}
\usepackage{stmaryrd}

\newcommand{\Q}{\mathbb{Q}}
\newcommand{\N}{\mathbb{N}}
\newcommand{\Z}{\mathbb{Z}}
\newcommand{\R}{\mathbb{R}}
\newcommand{\Primes}{\mathbb{P}}
\newcommand{\st}{\text{ s.t. }}
\newcommand{\txtand}{\text{ and }}
\newcommand{\txtor}{\text{ or }}
\newcommand{\lxor}{\veebar}


\begin{document}

	\maketitle
	
	\begin{abstract}
		Only the questions with a * are compulsory (but do all of them!).
	\end{abstract}
	
	\section{Fundamental Theorem of Calculus}
	\paragraph{Statement}\mbox{}\\
	Let $f$ be a continuous real-valued function defined on a closed interval $\left[ 0,x \right]$.
	Let $F$ be the function defined, for all $t \in \left[ 0,x \right]$, by $F(x) = \int_0^x f(t) dt$.
	
	Then $F$ is uniformly continuous on $\left[ 0,x \right]$ and differentiable on the open interval $\left( a,b \right)$, and $F'(x) = f(x)$ for all $x \in \left( a,b \right)$ so $F$ is an anti-derivative of $f$.
	
	\paragraph{Generalization / Corollary}\mbox{}\\
	Let \( f(x) \) be a continuous function on the closed interval \([a, b]\), and let \( F \) be an anti-derivative of \( f \).
	Prove that
	\[
	\int_a^b f(x) \, dx = F(b) - F(a).
	\]
	
	\paragraph{Application}\mbox{}\\
	Evaluate the following definite integral using the Fundamental Theorem of Calculus:
	\[
	\int_0^{\pi/2} \sin(x) \, dx
	\]
	
	Evaluate the following definite integral using the Fundamental Theorem of Calculus:
	\[
	\int_1^4 \frac{1}{x^2} \, dx
	\]
	
	
	\section{Integration Techniques}
	\paragraph{Reminder}
	$$\int f(u) \, du = \int f(g(x)) \cdot g'(x) \, dx \quad \text {or} \quad \int f(u) \, du = \int f(x) \cdot \frac{du}{dx} \, dx$$
	$$\int u v' = uv - \int v u'$$
	
	\paragraph{Substitution / Change of Variable}\mbox{}
	
	
	\textbf{Exercise 1:}
	Evaluate the following integral using the method of substitution:
	\[
	\int \frac{2x}{x^2 + 1} \, dx
	\]
	\textit{Hint:} Let \( u = x^2 + 1 \) and then find \( du \) to perform the substitution.
	
	\textbf{Exercise 2: (*)}
	Evaluate the following integral using the method of substitution:
	\[
	\int \frac{1}{\sqrt{1 - x^2}} \, dx
	\]
	\textit{Hint:} Let \( x = \sin(u) \) and then find \( du \) to perform the substitution.
	
	\textbf{Exercise 3:}
	Evaluate the following integral using a trigonometric substitution:
	\[
	\int \frac{1}{4 + x^2} \, dx
	\]
	\textit{Hint:} Use the substitution \(u = x/2\) to simplify the integral.
	
	
	\paragraph{Integration by Parts}\mbox{}
	
	\textbf{Exercise A:}
	Compute the following integral using integration by parts:
	\[
	\int x \ln(x) \, dx
	\]
	
	\textbf{Exercise B:}
	Find the value of the integral using integration by parts:
	\[
	\int x^2 e^x \, dx
	\]
	
	\textbf{Exercise C: (*)}
	Compute the following integral using integration by parts:
	\[
	\int x \cos(x) \, dx
	\]
	
	\textbf{Exercise D:}
	Evaluate the following integral using the method of substitution:
	\[
	\int e^{2x} \cos(2x) \, dx
	\]
	
	
	\paragraph{Further integration techniques}\mbox{}
	
	\textbf{Exercise $\alpha$:}
	Perform partial fraction decomposition on the following rational expression:
	\[
	\frac{3x^2 - 2x - 1}{x^3 - x^2 + x - 1}
	\]
	\textit{Hint:} Factor the denominator and express the given expression as a sum of simpler fractions.
	
	\textbf{Exercise $\beta$: (*)}
	Evaluate the following improper integral by decomposition:
	\[
	\int_0^{+\infty} e^{-x}\, dx
	\]
	\textit{Hint:} Evaluate the improper integral by considering the limits is $a$, and let $a$ approach infinity.
	
	\textbf{Exercise $\gamma$:}
	Approximate the value of the integral
	\[
	\int_0^{\pi/2} \sin(x) \, dx
	\]
	using the Trapezoidal Rule with \(n = 4\) sub-intervals.
	
	\textbf{Exercise $\delta$:}
	Estimate the value of the integral
	\[
	\int_0^{\pi/2} \sin(x) \, dx
	\]
	using Simpson's Rule with \(n = 3\) sub-intervals.
	
	
	\section{Applications}
	\paragraph{Areas between curves}\mbox{}\\
	Determine the area of the region enclosed by the curves \(y = \sin(x)\) and \(y = -\sin(x)\) over the interval \([0, \pi]\).\\
	\textit{Hint:} Begin by finding the points of intersection between the two curves within the given interval.
	Then, set up the integral to calculate the area between the curves.
	
	\paragraph{Volumes of revolution} (Disk Method)\\
	Find the volume of the solid generated by revolving the region bounded by \(y = x^2\) and the x-axis, over the interval \([0, 1]\), about the x-axis using the disk method.\\
	\textit{Hint:} Determine the limits of integration, the radius of the disks, and set up the integral to find the volume.
	
	\paragraph{Arc length of curves (*)}\mbox{}\\
	Find the arc length of the curve defined by \(y = \sqrt{x}\) over the interval \([1, 4]\).\\
	\textit{Hint:} Use the formula for arc length \(\int_a^b \sqrt{1 + (f'(x))^2} \, dx\) to calculate the arc length of the curve.
	
	\paragraph{Surface area of revolution}\mbox{}\\
	Determine the surface area of the solid generated by revolving the curve \(y = x^2\) over the interval \([0, 1]\) about the x-axis.\\
	\textit{Hint:} Use the formula for surface area of revolution \(\int_a^b 2\pi f(x) \sqrt{1 + (f'(x))^2} \, dx\) to calculate the surface area.
	
	

\end{document}
