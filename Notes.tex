
\emph{Math Refresher 2023}

This course teaches basic mathematical metho dologies for proofs. It is
intended for students with a lack of mathematical background, or with a
lack of confidence in mathematics. We will try to cover most of the
prerequisites of the courses in the master\textquotesingle s, i.e. basic
algebra/analysis and basic applications.

\hypertarget{presentation}{%
\section{Presentation}\label{presentation}}

\begin{itemize}
\tightlist
\item
  Paul Dubois
\item
  3rd year PhD @ Centrale / TheraPanacea
\item
  Research topic: AI applied to radiotherapy
\item
  Email:
  \href{mailto:b00795695@essec.edu}{\nolinkurl{b00795695@essec.edu}}
  (for any question)
\item
  Course structure

  \begin{itemize}
  \tightlist
  \item
    8*3h arranged as 1h20min lecture - 1/3h break - 1h20min lecture
  \item
    No pb class planned, but lectures will have integrated live
    exercises
  \item
    Interrupt if needed (do \emph{not} wait for the end of the lecture)
  \end{itemize}
\item
  Examination

  \begin{itemize}
  \tightlist
  \item
    The course is pass/fail
  \item
    Spoiler: All of you will pass
  \item
    Home exercises, you will need 80+\% to pass
  \item
    How long do you need to complete exercises (should take 30min to
    1h)?
  \item
    How many exercises do you want? (2-4?)
  \item
    Hand in paper of PDF? (vote)
  \item
    In the unlikely event of not passing, you will be able to do some
    extra work to pass
  \end{itemize}
\item
  Course notes are still under construction (as I will adjust according
  to the speed of the class); I will give it to you at the end of the
  course.
\item
  Final questions before we start?
\end{itemize}

\hypertarget{assumed-to-be-known}{%
\section{Assumed to be known}\label{assumed-to-be-known}}

\begin{itemize}
\tightlist
\item
  4 operations (+,-,*,/)
\item
  integer vs rational vs decimal
\item
  what is a prime number
\item
  basic (linear) equations solving
\end{itemize}

\hypertarget{sets}{%
\section{Sets}\label{sets}}

\begin{itemize}
\tightlist
\item
  sets of numbers (\(\N\), \(\Z\), \(\R\), \(\Q\), \(\Primes\))
\item
  complex sets (with \(\{ \}\))
\item
  examples (draw them):

  \begin{itemize}
  \tightlist
  \item
    \(\{n \mid 4<n<10, n \in \N \}\)
  \item
    \(\{2n-1 \mid 4<n<10 , n \in \N \}\)
  \item
    \(\{x \mid 4<x<10, x \in \R \}\)
  \item
    \(\{x \mid 4<x^2<10 \}\)
  \item
    \(\{(x,y) \mid 0<x<2 , 1<y<3, x \in \R, y \in \R \}\)
  \end{itemize}
\item
  live exercises: draw set + define set from drawing
\item
  intervals (\(\left[a,b\right]\) \& \(\left(a,b\right)\)); example:
  \(\left[-2, 3\right)\)
\item
  sets unions \& intersections
\item
  examples:

  \begin{itemize}
  \tightlist
  \item
    \(\left[0,1\right) \cup \left(2,3\right]\)
  \item
    \(\left(0,1\right) \cap \left[0.5,2\right]\)
  \item
    \(\left[-2,5\right) \cap \N\)
  \item
    \(\left[-2,5\right) \cap \Z\)
  \end{itemize}
\item
  live exercises:

  \begin{itemize}
  \tightlist
  \item
    compute and plot the inersection and union of
    \(A = \left(1, 5\right)\) and \(B = \left(3, 7\right]\).
  \item
    compute and plot the inersection and union of
    \(C = \left(-\infty, 2\right]\) and
    \(D = \left[0, +\infty \right)\).
  \end{itemize}
\item
  quantifiers: \(\forall\), \(\exists\)
\item
  exmaple (simple):

  \begin{itemize}
  \tightlist
  \item
    \(S = \{1,3,5,7,8\}\): \(\forall s \in S, \st \leq 10\)
  \item
    \(S = \{1,3,5,7,8\}\): \(\exists s \in \S \st s \text{ is pair}\)
  \end{itemize}
\item
  example (combined): "for any number, there is a (natural) number
  greater" (\(\forall x \in \R, \exists n \in \N s.t. n>x\))
\item
  live exercises:

  \begin{itemize}
  \tightlist
  \item
    \(S = \{5,6,3,1\}\) "all elements of \(S\) are positive"
  \item
    \(S = \{5,6,3,1\}\) "there is an odd element in \(S\)"
  \item
    \(S = \{5,6,3,1\}\) "there is an even element in \(S\) that is not a
    multiple of 4"
  \end{itemize}
\item
  implications \(\implies\), \(\impliedby\), \(\iff\)
\item
  examples:

  \begin{itemize}
  \tightlist
  \item
    \(x>1 \implies x \text{positive}\)
  \item
    \(k \in \Z \impliedby k \in \N\)
  \item
    \(k \in \Z  \txtand k\geq 0 \iff k \in \N\)
  \end{itemize}
\item
  live exercises:

  \begin{itemize}
  \tightlist
  \item
    "if \(x\) is positive, then it is the square of another number"
  \item
    "\(n\) is pair is equivalent to \(n=2m\) for some integer \(m\)"
  \end{itemize}
\item
  extreme values (\(\min\),\(\max\) vs \(\inf\),\(\sup\))
\item
  live exercises:

  \begin{itemize}
  \tightlist
  \item
    find the extreme values of the set \(A = \{x \in \R \mid x>0\}\).
  \item
    find the extreme values of the set
    \(B = \{1-\frac{1}{n} \mid n\in \N\}\).
  \end{itemize}
\end{itemize}

\hypertarget{boolean-algebra}{%
\section{Boolean Algebra}\label{boolean-algebra}}

\begin{itemize}
\tightlist
\item
  principle (only \(0\) and \(1\))
\item
  \(+\) and \(*\) for booleans: \(\lor\) and \(\land\)
\item
  \(not\) (\(\lnot\))
\item
  tables
\item
  De Morgan\textquotesingle s law
  (\(\lnot (a \land b) = \lnot a \lor \lnot b\) and
  \(\lnot (a \lor b) = \lnot a \land \lnot b\))
\item
  \(implications\) operators (\(\implies, \impliedby, \iff\));
  \(\text{x}or\) operator (\(\lxor\))
\item
  live exercise:

  \begin{itemize}
  \tightlist
  \item
    express \(\lxor\) in terms of \(\lor, \land, \lnot\)
  \item
    express \(\implies\) in terms of \(\lor, \land, \lnot\)
  \item
    express \(\land\) in terms of \(\lor, \lnot\)
  \item
    express \(\lor\) in terms of \(\land, \lnot\)
  \end{itemize}
\end{itemize}

\hypertarget{modular-arithmetic}{%
\section{Modular arithmetic}\label{modular-arithmetic}}

\begin{itemize}
\tightlist
\item
  Euclidean division of \(a\) by \(b\) (\(a=bk+r\) with
  \(0 \leq r < b\))
\item
  example with \(a=35\), \(b=2,3,4,5,6,7,8\)
\item
  modular classes
  (\(12 \equiv 7 \equiv 22 \equiv 102 \equiv -3 \equiv -103 \mod 5\)
  i.e. \(\{2+5k \mid k \in \Z \}\))
\item
  live exercises:

  \begin{itemize}
  \tightlist
  \item
    give 3 numbers that are congruent to 3 mod 7
  \item
    give a test in terms of modular arithmetic that is equivalent to
    "\(n\) is odd"
  \item
    give a test in terms of modular arithmetic that is equivalent to
    "\(n\) is a nultiple of \(k\)" (for \(k\) a natural number greater
    than two)
  \item
    what does it mean for \(n\) to say that \(n \equiv 5 \mod 10\)?
  \item
    find the least positive value of \(x\) such that
    \(71 \equiv x \mod 8\)
  \end{itemize}
\item
  modular operations (+,-,* \(\mod n\))
\item
  GCD and \(\square^{-1} \mod p\)
\item
  example:

  \begin{itemize}
  \tightlist
  \item
    compute the GCD of \(270\) and \(192\) (answer: \(6\))
  \item
    compute \(5^{-1} \mod 11\)
  \end{itemize}
\item
  live exercises:

  \begin{itemize}
  \tightlist
  \item
    find the least positive value of \(x\) such that
    \(89 \equiv (x + 3) \mod 4\)
  \item
    what is \(x \mod 10\) if \(96 \equiv x / 7 \mod 5\)
  \item
    find an \(x\) such that \(5x \equiv 4 \mod 11\)
  \item
    if \(x\) is congruent to \(13 \mod 17\) then \(7x - 3\) is congruent
    to which number \(\mod 17\)?
  \end{itemize}
\end{itemize}

\hypertarget{functions}{%
\section{Functions}\label{functions}}

\begin{itemize}
\tightlist
\item
  functions def
\item
  image vs pre-image
\item
  span vs kernel
\item
  examples:

  \begin{itemize}
  \tightlist
  \item
    \(f: x \to 3x+1\)
  \item
    \(g: x \to x^2-1\)
  \item
    \(h: x \to 8\)
  \end{itemize}
\item
  live exercises:

  \begin{itemize}
  \tightlist
  \item
    compute the image of \(2\) by \(f(x) = \frac{(x+1)^2 - x}{x-3}\)
  \item
    compute the preimage(s) of \(5\) by \(f(x) = 2x-3\)
  \item
    compute the kernel of \(f(x) = -3x+2\)
  \item
    compute the span of \(f(x) = 5-(2x)^4\)
  \end{itemize}
\item
  typical plotting of functions: set of points \((x,y)\) s.t.
  \(y = f(x)\)
\end{itemize}

\hypertarget{sequences}{%
\section{Sequences}\label{sequences}}

\begin{itemize}
\tightlist
\item
  sequences def: general formula
\item
  example: \(u_n = n^3-5n^2\)
\item
  sequences def: recursive formula
\item
  example: \(u_0 = 5, u_{n+1} = u_n^2-u_n+2\)
\item
  live exercises:

  \begin{itemize}
  \tightlist
  \item
    consider the (arithmetic) sequence \(\{a_n\}\) defined by
    \(a_{n+1}=a_n+2\) and \(a_0=-1\):

    \begin{itemize}
    \tightlist
    \item
      find the first five terms of the sequence
    \item
      find the common difference between consecutive terms
    \item
      find a formula for \(a_n\) (without using \(a_{n-1}\))
    \end{itemize}
  \item
    consider the (geometric) sequence \(\{b_n\}\) defined by
    \(b_n=3*2^n\)

    \begin{itemize}
    \tightlist
    \item
      find the first five terms of the sequence
    \item
      find the common ratio between consecutive terms
    \item
      find a formula for \(b_{n+1}\) (using only \(b_n\), no \(n\))
    \end{itemize}
  \end{itemize}
\end{itemize}

\hypertarget{essence-of-proofs}{%
\section{Essence of proofs}\label{essence-of-proofs}}

\begin{itemize}
\tightlist
\item
  proof: assumption =\textgreater{} conclusion
\item
  direct with \(n \geq 0 \implies 2n \geq 4n\)
\item
  cases split with \(n \equiv n^2 \mod 2\)
\item
  contradiction with \(\sqrt{2} \not \in \Q\)
\item
  induction with \(u_0 = 2, u_{n+1} = \frac{u_n+1}{2} \implies u_n>1\)
\item
  live exercises:

  \begin{itemize}
  \tightlist
  \item
    prove that for all real numbers \(x\), if \(x\) is positive, then
    \(x^3\) is also positive
  \item
    prove that the square root of 3 is irrational, i.e., it cannot be
    expressed as a fraction of two integers.
  \item
    prove by mathematical induction that for all non-negative integers
    \(n\), \(3^n - 1\) is divisible by \(2\).
  \item
    use mathematical induction to prove that for all positive integers
    \(n\), the sum of the first \(n\) odd integers is given by the
    formula: \(1 + 3 + 5 + ... + (2n - 1)\) is \(n^2\).
  \end{itemize}
\end{itemize}

\hypertarget{asymptotic-analysis}{%
\section{Asymptotic analysis}\label{asymptotic-analysis}}

\begin{itemize}
\tightlist
\item
  definition (\(\varepsilon, \delta\))
\item
  examples / live exercises:

  \begin{itemize}
  \tightlist
  \item
    prove that limit of \(u_n = \frac{n^2+1}{n^2}\) as \(n \to +\infty\)
    is \(1\)
  \item
    prove that limit of \(f(x) = \frac{2x-1}{x}\) as \(x \to -\infty\)
    is \(2\)
  \item
    prove that limit of \(u_n = \frac{1}{\sqrt{n}}\) as
    \(n \to +\infty\) is \(0\)
  \item
    prove that \(u_n = 2n^3\) diverges to \(+\infty\) as
    \(n \to +\infty\)
  \item
    prove that limit of \(f(x) = \frac{1}{x^2}\) as \(x \to 0\) is
    \(+\infty\)
  \item
    prove that limit of \(f(x) = \frac{1}{x}\) as \(x \to 0^-\) is
    \(-\infty\)
  \end{itemize}
\item
  operations on limits (\(+\), \(-\), \(*\), and \(/\))
\item
  live exercises:

  \begin{itemize}
  \tightlist
  \item
    calculate
    \(\lim_{n\to \infty} \left(2+\frac{-1}{2n}\right)\left(3-\frac{4}{-n^2}\right)+5\)
  \item
    calculate \(\lim_{n\to \infty} \frac{-2n+1}{8n}\)
  \item
    calculate \(\lim_{x\to \infty} \frac{3x^2 + 2x}{4x^2 - 1}\)
  \item
    determine the behaviour of \(u_n = (-2)^n\) as \(n \to +\infty\)
  \end{itemize}
\end{itemize}

\hypertarget{large-operators}{%
\section{Large operators}\label{large-operators}}

\begin{itemize}
\tightlist
\item
  \(\sum\), \(\prod\), \(\bigcup\), \(\bigcap\)
\item
  examples:

  \begin{itemize}
  \tightlist
  \item
    "product of numbers from 10 to 20"
  \item
    "sum of squares up to 10"
  \item
    \(\bigcup_{x \in \{1,4,10.5, 21.75\}} \left[ x-0.5, x+0.5 \right]\)
  \item
    \(\bigcap_{n \in \N^*} \left[ -\frac{1}{n}, \frac{1}{n} \right]\)
  \end{itemize}
\item
  live exercises:

  \begin{itemize}
  \tightlist
  \item
    what set does the last example corresponds to?
  \item
    define the factorial
  \item
    give an expression for the sum of inverses from \(1\) to \(1000\)
  \item
    give an expression for the product of all prime numbers smaller than
    \(10000\)
  \item
    give an expression for the sum of factorials from \(100\) to \(200\)
  \end{itemize}
\end{itemize}

\hypertarget{series}{%
\section{Series}\label{series}}

\begin{itemize}
\tightlist
\item
  definition: sum of a sequence
\item
  partial sums: \(S_n = \sum_{k=0}^n u_k\)
\item
  examples:

  \begin{itemize}
  \tightlist
  \item
    \(S_n = \sum_{k=0}^n k^2\)
  \item
    \(S_n = \sum_{k=0}^n \frac{1}{k!}\)
  \item
    \(S_n = \sum_{k=0}^n \frac{1}{2^k}\)
  \end{itemize}
\item
  popular series:

  \begin{itemize}
  \tightlist
  \item
    geometric series
  \item
    harmonic series
  \item
    alternating series
  \end{itemize}
\item
  convergence: if the sequence of partial sums converges
\item
  convergence tests:

  \begin{itemize}
  \tightlist
  \item
    comparison test
  \item
    integral test (see later)
  \item
    ratio test
  \item
    root test
  \item
    alternating series test
  \end{itemize}
\item
  live exercises:

  \begin{itemize}
  \tightlist
  \item
    prove that the series \(\sum_{k \in \N} \frac{1}{k}-\frac{1}{k+1}\)
    converges
  \item
    prove that the series \(\sum_{k \in \N} \frac{1}{k!}\) converges
  \item
    prove that the series \(\sum_{k \in \N} \frac{1}{2^k}\) converges
  \item
    prove that the series \(\sum_{k \in \N} \frac{1}{k}\) diverges
  \item
    prove that the series \(\sum_{k \in \N} \frac{1}{k^2}\) converges
  \item
    prove that the series \(\sum_{k \in \N} \frac{k^10}{2^k}\) converges
  \end{itemize}
\end{itemize}

\hypertarget{affine-functions}{%
\section{Affine functions}\label{affine-functions}}

\begin{itemize}
\tightlist
\item
  definition: \(f(x) = ax+b\) (\(a\) is the slope, \(b\) is the
  intercept)
\item
  examples:

  \begin{itemize}
  \tightlist
  \item
    \(f(x) = 2x+1\)
  \item
    \(f(x) = -3x+2\)
  \item
    \(f(x) = 5\)
  \end{itemize}
\item
  live exercises:

  \begin{itemize}
  \tightlist
  \item
    plot the function \(f(x) = 2x+1\)
  \item
    plot the function \(f(x) = -3x+2\)
  \item
    find the affine function that passes through the points \((1,2)\)
    and \((3,4)\)
  \end{itemize}
\item
  parallel (same slope) and orthogonal lines (negative reciprocal slope)
\item
  live exercises:

  \begin{itemize}
  \tightlist
  \item
    find the equation of the line parallel to \(y=2x+1\) that passes
    through \((5,3)\)
  \item
    find the equation of the line orthogonal to \(y=2x+1\) that passes
    through \((8,7)\)
  \end{itemize}
\end{itemize}

\hypertarget{quadratic-functions--equations}{%
\section{Quadratic functions /
equations}\label{quadratic-functions--equations}}

\begin{itemize}
\tightlist
\item
  definition: \(f(x) = ax^2+bx+c\) (\(a\) is the quadratic coefficient,
  \(b\) is the linear coefficient, \(c\) is the constant)
\item
  example: \(f(x) = x^2+3\) (plot it)
\item
  solving quadratic equations (do demo)
\item
  3 forms of quadratic functions:

  \begin{itemize}
  \tightlist
  \item
    \(f(x) = a(x-x_1)(x-x_2)\)
  \item
    \(f(x) = ax^2+bx+c\)
  \item
    \(f(x) = a(x-x_0)^2+y_0\)
  \end{itemize}
\end{itemize}

TODO:

\begin{itemize}
\tightlist
\item
  Graph of usual functions
\item
  Derivatives
\item
  Usual functions (sin, cos, tan, exp, log)
\item
  Integration
\item
  Complex numbers
\item
  Vectors (concept, sum, scalar product)
\item
  Equations for lines (2D, 3D) and planes (3D)
\item
  Matrices (concept, sum, product)
\item
  Mutli-dimensional functions
\item
  Inversing matrices (+ row reduction; span)
\item
  Linear regression
\item
\end{itemize}
