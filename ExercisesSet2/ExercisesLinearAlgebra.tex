\documentclass[]{article}

%opening
\title{Exercises Set 2}
\author{Paul Dubois}

\usepackage{amsmath}
\usepackage{amsfonts}
\usepackage{amsthm}
\usepackage{amssymb}
\usepackage{mathrsfs}
\usepackage{stmaryrd}

\newcommand{\Q}{\mathbb{Q}}
\newcommand{\N}{\mathbb{N}}
\newcommand{\Z}{\mathbb{Z}}
\newcommand{\R}{\mathbb{R}}
\newcommand{\Primes}{\mathbb{P}}
\newcommand{\st}{\text{ s.t. }}
\newcommand{\txtand}{\text{ and }}
\newcommand{\txtor}{\text{ or }}
\newcommand{\lxor}{\veebar}


\begin{document}
	
	\maketitle
	
	\begin{abstract}
		Only the questions with a * are compulsory (but do all of them!).
	\end{abstract}
	
	\section{Systems of Linear Equations}
	\paragraph{Reduced Row Echelon Form}
	Find the Reduced Row Echelon Form of the following matrix:
	$$\begin{pmatrix}
		1 & 2 & 3 & 4\\
		0 & 6 & 2 & 3\\
		0 & 0 & 0 & 10
	\end{pmatrix}$$
	
	\paragraph{Gaussian Elimination}
	(*) Solve the following linear system using Gaussian Elimination:
	\begin{align*}
		2x + y - z &= 4 \\
		3x + 2y + z &= 5 \\
		x - y + 3z &= 7
	\end{align*}
	Start by writing the augmented matrix for the system and perform the necessary row operations to find the solution.
	
	
	\section{Vector Spaces}
	\paragraph{Linear Independence}
	Consider the following vectors in $\mathbb{R}^3$:
	$$
	\mathbf{v_1} = \begin{bmatrix}
		1 \\
		2 \\
		0
	\end{bmatrix}, \quad
	\mathbf{v_2} = \begin{bmatrix}
		0 \\
		1 \\
		1
	\end{bmatrix}, \quad
	\mathbf{v_3} = \begin{bmatrix}
		3 \\
		5 \\
		2
	\end{bmatrix}
	$$
	Show that $\mathbf{v_1}$, $\mathbf{v_2}$, and $\mathbf{v_3}$ are linearly independent.
	
	\paragraph{Space of Polynomials}
	Let $\mathbb{P}_2$ be the space of polynomials of degree at most 2.
	Consider the following polynomials:
	$$p_1(x) = 1, \quad p_2(x) = 2x, \quad p_3(x) = 3x^2 - 1$$
	Show that the polynomials $p_1(x)$, $p_2(x)$, and $p_3(x)$ form a spanning set for $\mathbb{P}_2$.\\
	Express an arbitrary polynomial $q(x) \in \mathbb{P}_2$ as a linear combination of $p_1(x)$, $p_2(x)$, and $p_3(x)$.
	
	
	\section{Matrix Inverses}
	\paragraph{2x2 Matrices}
	Let
	$$A = \begin{bmatrix}
		2 & 1 \\
		3 & 2
	\end{bmatrix}$$
	Determine whether matrix $A$ is invertible.
	If it is, find its inverse $A^{-1}$.
	Verify your result by multiplying $A$ by its inverse $A^{-1}$ and showing that you get the identity matrix.
	
	\paragraph{3x3 Matrices}
	(*) Let
	$$B = \begin{bmatrix}
		2 & 1 & 3 \\
		1 & 2 & 0 \\
		-1 & 3 & 1
	\end{bmatrix}$$
	Determine whether matrix $B$ is invertible. If it is, find its inverse $B^{-1}$.
	Verify your result by multiplying $B$ by its inverse $B^{-1}$ and showing that you get the identity matrix.
	
	
	\section{Eigenvalues and Eigenvectors}
	\paragraph{Basic 2x2 Case}
	(*) Consider the matrix
	$$A = \begin{bmatrix}
		3 & 1 \\
		1 & 3
	\end{bmatrix}$$
	Find the eigenvalues of matrix $A$.\\
	For each eigenvalue, find the corresponding eigenvector.
	
	\paragraph{Repeated Eigenvalues}
	Consider the matrix
	$$B = \begin{bmatrix}
		1 & -1 \\
		0 & 1
	\end{bmatrix}$$
	Find the eigenvalues of matrix $B$.\\
	For each eigenvalue, find the corresponding eigenvector.
	
	\paragraph{Basic 3x3 Case}
	Consider the matrix
	$$C = \begin{bmatrix}
		1 & 2 & 0 \\
		0 & 2 & 0 \\
		0 & 1 & 3
	\end{bmatrix}$$
	Find the eigenvalues of matrix $C$.\\
	For each eigenvalue, find the corresponding eigenvector.
	
	
	\section{Diagonalization}
	For each matrix from the "Eigenvalues and Eigenvectors" section,  determine whether matrix is diagonalizable.
	If it is, diagonalize it by finding a diagonal matrix $D$ and an invertible matrix $P$ such that $A = PDP^{-1}$.
	
	\section{Orthogonal Vectors}
	\paragraph{Orthogonality}
	Given the vectors
	$$
	\mathbf{u} = \begin{bmatrix}
		2 \\
		-1 \\
		0
	\end{bmatrix}, \quad
	\mathbf{v} = \begin{bmatrix}
		1 \\
		2 \\
		1
	\end{bmatrix}, \quad
	\mathbf{w} = \begin{bmatrix}
		0 \\
		1 \\
		-2
	\end{bmatrix}
	$$
	Determine which of the vectors $\mathbf{u}$, $\mathbf{v}$, and $\mathbf{w}$ are orthogonal to each other.
	
	
	\paragraph{Gram-Schmidt Orthogonalization}
	(*) Given the vectors
	$$
	\mathbf{v}_1 = \begin{bmatrix}
		1 \\
		2 \\
		0
	\end{bmatrix}, \quad
	\mathbf{v}_2 = \begin{bmatrix}
		1 \\
		1 \\
		1
	\end{bmatrix}, \quad
	\mathbf{v}_3 = \begin{bmatrix}
		0 \\
		1 \\
		2
	\end{bmatrix}
	$$
	Apply the Gram-Schmidt orthogonalization process to find an orthonormal basis for the subspace spanned by these vectors.\\
	Verify that the resulting vectors are indeed orthogonal and normalized (by computing their norm and pairwise dot product).
	
	
\end{document}
