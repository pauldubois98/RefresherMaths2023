\documentclass[]{article}

%opening
\title{Exercises Set 1}
\author{Paul Dubois}

\usepackage{amsmath}
\usepackage{amsfonts}
\usepackage{amsthm}
\usepackage{amssymb}
\usepackage{mathrsfs}
\usepackage{stmaryrd}

\newcommand{\Q}{\mathbb{Q}}
\newcommand{\N}{\mathbb{N}}
\newcommand{\Z}{\mathbb{Z}}
\newcommand{\R}{\mathbb{R}}
\newcommand{\Primes}{\mathbb{P}}
\newcommand{\st}{\text{ s.t. }}
\newcommand{\txtand}{\text{ and }}
\newcommand{\txtor}{\text{ or }}
\newcommand{\lxor}{\veebar}


\begin{document}
	
	\maketitle
	
	\begin{abstract}
		Only the questions with a * are compulsory (but do all of them!).
	\end{abstract}
	
	\section{Optimization}
	\subsection{One dimension}
	Let's consider $f(x) = x^2-x+3$.
	We are interested in finding the minimum of this function; 
	i.e. we want to find $x^* \in \R$ such that $f(x^*)$ is the smallest possible, mathematically:
	$$\forall x\in \R, f(x) \geq f(x^*)$$
	\textit{It turns out that since this function is quadratic, you can manually compute that $x^*=0.5$.
		However, we will suppose that this is a complex function, for which we cannot find the minimum by hand.
		The techniques we will develop should be generalization to more complex functions, for which you won't be able to find the minimum "by hand".}
	
	\paragraph{Method 1}
	First, let's try to brute force our problem: compute $f(x)$ for $x=-1.75$, $x=-0.75$, $x=-0.25$, $x=0.25$, $x=0.75$, $x=1.25$, $x=-1.75$, $x=2.25$, $x=2.75$, and $x=3.25$.\\
	Find the $x$ value that gave you the smallest value for $f(x)$.
	This will be you first "guess" for $x^*$.\\
	\textbf{This technique is called "grid search".}
	
	
	
	
	
	
	
\end{document}
